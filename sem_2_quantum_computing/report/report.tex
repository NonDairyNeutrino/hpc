\documentclass{article}
% \renewcommand{\thesection}{\arabic{section}}
% \setcounter{tocdepth}{1}
% index
% \usepackage{imakeidx}
% \makeindex%[title=Index, intoc]
% glossary
% \usepackage{glossaries}
% \makeglossaries
% geometry formatting
\usepackage{authblk}
\usepackage[utf8]{inputenc}
\usepackage[margin=0.75in]{geometry}
\usepackage{parskip, setspace}
\setstretch{1.15}
% math formatting
\usepackage{amsmath, amsfonts, braket}
% \numberwithin{equation}{section}
% rich text
\usepackage{graphicx, caption}
\usepackage{hyperref}
% \usepackage{xcolor}
\hypersetup{
    colorlinks=true,
    linkcolor=black,  
    urlcolor=blue,
    citecolor=blue,
    pdftitle={A Survey of Computational Physics},
    pdfpagemode=FullScreen,
}
% bibliography
\usepackage{biblatex}
\addbibresource{bib.bib}

% \renewcommand{\arraystretch}{1.5} % give a little more space in table

\title{CS 530: High-Performance Computing \\ Seminar 2: Quantum Computing}
\author{Nathan Chapman}
\affil{Department of Computer Science \\ Central Washington University}
\date{\today}

\begin{document}

\maketitle

\tableofcontents

\section{History of Quantum Computation \& Information}

\section{Quantum Bits}

    \begin{itemize}
        \item The bit and qubit is the most fundamental concept of information
        \item A classical bit has a state: either 0 or 1
        \item A quantum bit   has a state: $\ket{0}, \ket{1}, \alpha \ket{0} + \beta \ket{1}$ for complex $\alpha, \beta$ such that $|\alpha|^2 + |\beta|^2 = 1$
        \item The state of a qubit is a unit vector in a two-dimensional complex vector space.  In other words, qubits similar to are unit quarternions.
        \item $\ket{0}, \ket{1}$ are orthonormal and form computational basis states
        \item Can't directly measure $\alpha, \beta$
        \item Example: a ``quantim coin'' with state $\ket{+} = \frac{1}{\sqrt{2}} \ket{0} + \frac{1}{\sqrt{2}} \ket{1}$ and 50-50 probability
        \item Can write $\ket{\psi} = e^{i \gamma} \left(\cos \left( \frac{\theta}{2} \right) \ket{0} + e^{i \phi} \sin \left( \frac{\theta}{2} \right) \ket{0} \right)$
        \item Because $e^{i \gamma}$ has no observable effect, we can reduced the above to $\ket{\psi} = \cos \left( \frac{\theta}{2} \right) \ket{0} + e^{i \phi} \sin \left( \frac{\theta}{2} \right) \ket{0}$
        \item Finished page 15 at Bloch sphere
    \end{itemize}

\section{Quantum Computation}

    \subsection{Quantum Gates}

    \subsection{Quantum Circuits}

    \subsection{Examples}

        \subsubsection{Bell States}

        \subsubsection{Quantum Teleportation}

\section{Quantum Algorithms}

    \subsection{Examples}

        \subsubsection{The Quantum Fourier Transform}

        \subsubsection{The Quantum Search Algorithm}

\section{Quantum Information}

    \subsection{Quantum Cryptography}

\newpage
\nocite{*}
\printbibliography

\end{document}
\documentclass{report}
% \renewcommand{\thesection}{\arabic{section}}
\setcounter{tocdepth}{1}
% geometry formatting
\usepackage{authblk}
\usepackage[utf8]{inputenc}
\usepackage[margin=0.625in]{geometry}
\usepackage{parskip, setspace}
\setstretch{1.5}
% math formatting
\usepackage{amsmath, amsfonts}
% \numberwithin{equation}{section}
% code formatting
\usepackage{listings}
\usepackage{xcolor}

\definecolor{codegreen}{rgb}{0,0.6,0}
\definecolor{codegray}{rgb}{0.5,0.5,0.5}
\definecolor{codepurple}{rgb}{0.58,0,0.82}
\definecolor{backcolour}{rgb}{0.75,0.75,0.75}

\lstdefinestyle{mystyle}{
    backgroundcolor=\color{backcolour},   
    commentstyle=\color{codegreen},
    keywordstyle=\color{red},
    numberstyle=\tiny\color{codegray},
    stringstyle=\color{codepurple},
    basicstyle=\ttfamily\footnotesize,
    breakatwhitespace=false,         
    breaklines=true,                 
    captionpos=b,                    
    keepspaces=true,                 
    numbers=left,                    
    numbersep=5pt,                  
    showspaces=false,                
    showstringspaces=false,
    showtabs=false,                  
    tabsize=4
}

\lstset{style=mystyle}
% rich text
\usepackage{graphicx, caption, wrapfig}
\usepackage{hyperref}
% bibliography
\usepackage{biblatex}
\addbibresource{bib.bib}

% \renewcommand{\arraystretch}{1.5} % give a little more space in table

\title{CS 530: High-Performance Computing \\ Seminar 1: A Survey of Computational Physics}
\author{Nathan Chapman}
\affil{Department of Computer Science, Central Washington University}
\date{\today}

\begin{document}

\maketitle

\tableofcontents
\pagebreak

% \chapter{Introduction}

%     % \begin{itemize}
%     %     \item There are really hard and complex problems that are reasonably solved by hand
%     %     \item There are problems with no analytic solution
%     %     \item There are really big problems that require high-performance computing
%     %     \item Computational physics lies at the intersection of Math, Physics, and Computer Science
%     %     \item The difference between ``doing physics on the computer'' and ``computational physics'' is the consideration of numerical analysis
%     % \end{itemize}

% \chapter{Mathematical Methods}

%     \chapter{Taylor Series}

%     \chapter{Fourier Analysis}

%         \section{Fourier Series}

%         \section{The Fourier Transform}

%     \chapter{Properties of Differential Equations}

%         \section{Order}

%         \section{Linearity}

%         \section{Homogeneity}

%         \section{Chaos \& Entropy}

%             \subsection{Poincar{\'e} Sections}

%     \chapter{Ordinary Differential Equations}

%     \chapter{Partial Differential Equations}

%         \section{Parabolic PDEs}

%             \subsection{The Heat Equation}

%         \section{Hyperbolic PDEs}

%             \subsection{The Wave Equation}

%         \section{Sturm-Liouville Theory}

%         \section{Green's Functions}

%     \chapter{Differential Equations with Conditions}

%         \section{Initial Conditions}

%         \section{Boundary Conditions}

%             % \begin{itemize}
%             %     \item Dirichlet
%             %     \item Neumann
%             %     \item Robin
%             %     \item Mixed
%             %     \item Cauchy
%             % \end{itemize}

%     \chapter{Systems of Differential Equations}

%         \section{Coupled Systems}

%     \chapter{Eigenvalue Problems}

%     \chapter{Optimization}

% \chapter{Computational Methods}

%     \chapter{Programming Language}

%         \section{Fortran}

%         \section{Julia}

%         \section{Python}

%         \section{Matlab \& Mathematica}

%     \chapter{Finite-Differences}

%         \section{Order}

%         \section{Accuracy}

%     \chapter{Runge-Kutta Methods}

%     \chapter{Symplectic Integrators}

%     \chapter{Parareal - Parallel-in-Time Integration}

%     \chapter{The Fast Fourier Transform}

%         \section{FFTW}

%             \subsection{CUDA FFTW}

%     \chapter{Finite-Element Methods}

%     \chapter{Markov Chains \& Monte-Carlo Methods}

% \chapter{Physical Context}

%     \chapter{Orbital Dynamics}

%         \section{The N-Body Problem}

%     \chapter{Fluid Dynamics}

%         \section{The Lattice Boltzmann Method}

%     \chapter{Electrodynamics \& Magnetohydrodynamics}

%         \section{Fringing Electric Fields of Non-Ideal Capacitors}

%     \chapter{Many-Body Quantum Mechanics}

%         \section{Quadrupole-Quadrupole Interactions in a BEC}

%     \chapter{Numerical Relativity}

%         \section{Mercury's Perihelion Shift}

%             \subsection{Effects of Eccentricity}

%             \subsection{Gravitational Wave Chirp}

%         \section{Gravitational Waves}

%     \chapter{Chaotic Systems}

%         \section{Atmospheric Physics}

%         \section{Forced Oscillators}

%     \chapter{Honorable Mentions}

%         \section{Projectile Motion with Drag}

% \chapter{Numerical Analysis}
    
%     \chapter{Error}

%     \chapter{Lax Equivalence Theorem}

%     \chapter{Courant-Friedrichs-Lewy condition}

%     \chapter{Von Neumann Stability}

%     \chapter{Energy Drift}

%     \chapter{Stiff Differential Equations}

%     \chapter{Grids \& Meshes}

%     \chapter{Numerical Diffusion}

\chapter{Introduction}

    Each of the five pillars of physics (Classical Mechanicsm, Electromagnetism, Relativity, Quantum Mechanics, and Thermodynamics) has well-posed problems for which the mathematical models are either unreasonable or impossible to solve by hand.  As such, computational methods are needed to get an approximate solution.  
    
    There are many numerical methods to support the evolution of physical models for each domain.  Whether it be using a fourth-order Runge-Kutta method to solve the first-order ordinary differential equation that models a ballistic object under the influence of gravity and air-resistance, or using an eigth-order symplectic Yoshida integrator to model a system of bodies orbiting each other while accounting for the produced gravitational waves, there are methods for every physical context.  Though, because these methods are using approximations to their analytic counterparts in calculus, there is error that needs to be considered.

    Numerical analysis is the practice of tracking and rigorously accounting for the discrepancies that arise when approximating analytic mathematical methods by their finite and counterparts.  These errors can arise not only from approximating descriptions from calculus, but also from operations in linear algebra as they're applied to finding the eigenvalues of quantum operators.

\chapter{Classical Mechanics} \label{sec:classical}

    The ``everyday'' world as we know it is described by a model of physics that has been studied for millenia.  As such, we call this model of the behavior of nature ``Classical Mechanics''.  Now I hear you ask ``Why?''.  Well ``mechanics'' is the study of motion, and we include the ``classical'' preface to distinguish it from ``modern'' physics which encompasses all the physics discovered after around 1905 or 1925 (depending on who you ask); those physics will be covered later in sections \ref{sec:quantum} and \ref{sec:relativity}.  For now we take a look at the mathematical model that describes the motion of not-too-small, not-too-large objects moving at slow velocities.
    
    The motion of ``classical'' objects, ranging in scale from biological cells, sand, ants, birds, cars, airplanes, planets, up to stars, can described as the solution to what's called the \emph{equation of motion}.  The equation of motion for an object with mass $m$ under the influence of forces $F_i$ is determined by Newton's 2nd law of motion
    
    \begin{equation} \label{eq:newton}
        \sum_i \vec{F}_i = m \ddot{\vec{x}},
    \end{equation}

    where $\ddot{\vec{x}}$ represents the second time derivative of the position $\vec{x}$ known as \emph{acceleration}.  These forces completely determine the how objects move through space in time.
    
    Mathematically, the resulting equation is a second-order ordinary differential equation.  In many cases, this differential equation is homogenous to represent no time-dependent external forces supplying the physical system with energy.  Likewise, sometimes these forces can be represented by nonlinear terms; making the equation of motion much more complex.
    
    The following is a survey of topics in classical mechanics whose forces yield an equation of motion either too complex or too unreasonable to solve by hand.

\pagebreak

    \section{N-Body Orbits} \label{subsec:orbits}

        While so-called N-Body problems are prolific in physics and other sciences, a quintessential example is the gravitational interaction of astrophysical bodies i.e. massive bodies in space.

        \subsection{Mathematical Model: Systems of Ordinary Differential Equations}

            In the case of multi-body gravitaitonal interactions, the governing equation of motion is determined via Newton's law of gravitation between two objects with masses $m_i$ and $m_j$

            \begin{equation}
                F_{ij} = \frac{G m_i m_j}{r_{ij}^2} \hat{r}
            \end{equation}

            where $G = 6.67 \times 10^{-11} \frac{N m^2}{kg^2}$ is the universal gravitaional constant, $r_{ij}$ is the distance between the two objects, and $\hat{r}$ is the unit vector pointing from object $i$ toward $j$.  Not only are is there an equation of motion for each object, but each equation is coupled to every other equation.  This results in the three-dimensional vector differential equation

            \begin{equation}
                \sum_{j; i \neq j} \frac{G m_i m_j}{(x_i - x_j)^2 + (y_i - y_j)^2 + (z_i - z_j)^2} \frac{(x_j - x_i) \hat{x} + (y_j - y_i) \hat{y} + (z_j - z_i) \hat{z}}{\sqrt{(x_i - x_j)^2 + (y_i - y_j)^2 + (z_i - z_j)^2}} = m_i (\ddot{x_i} \hat{x} + \ddot{y_i} \hat{y} + \ddot{z_i} \hat{z})
            \end{equation}

            for object $i$, and a similar equation for every other object.  For example, a system of 3 objects interacting only via gravitational forces has the equations of motion 

            \begin{align}
                \frac{G m_2}{((x_1 - x_2)^2 + (y_1 - y_2)^2 + (z_1 - z_2)^2)^{3/2}} 
                \begin{bmatrix}
                    x_2 - x_1 \\
                    y_2 - y_1 \\
                    z_2 - z_1
                \end{bmatrix} + 
                \frac{G m_3}{((x_1 - x_3)^2 + (y_1 - y_3)^2 + (z_1 - z_3)^2)^{3/2}} 
                \begin{bmatrix}
                    x_3 - x_1 \\
                    y_3 - y_1 \\
                    z_3 - z_1
                \end{bmatrix} &= \begin{bmatrix}
                    \ddot{x_1} \\
                    \ddot{y_1} \\ 
                    \ddot{z_1}
                \end{bmatrix} \\
                %
                \frac{G m_1}{((x_2 - x_1)^2 + (y_2 - y_1)^2 + (z_2 - z_1)^2)^{3/2}} 
                \begin{bmatrix}
                    x_1 - x_2 \\
                    y_1 - y_2 \\
                    z_1 - z_2
                \end{bmatrix} + 
                \frac{G m_3}{((x_2 - x_3)^2 + (y_2 - y_3)^2 + (z_2 - z_3)^2)^{3/2}} 
                \begin{bmatrix}
                    x_3 - x_2 \\
                    y_3 - y_2 \\
                    z_3 - z_2
                \end{bmatrix} &= \begin{bmatrix}
                    \ddot{x_2} \\
                    \ddot{y_2} \\ 
                    \ddot{z_2}
                \end{bmatrix} \\
                %
                \frac{G m_1}{((x_3 - x_1)^2 + (y_3 - y_1)^2 + (z_3 - z_1)^2)^{3/2}} 
                \begin{bmatrix}
                    x_1 - x_3 \\
                    y_1 - y_3 \\
                    z_1 - z_3
                \end{bmatrix} + 
                \frac{G m_2}{((x_3 - x_2)^2 + (y_3 - y_2)^2 + (z_3 - z_2)^2)^{3/2}} 
                \begin{bmatrix}
                    x_2 - x_3 \\
                    y_2 - y_3 \\
                    z_2 - z_3
                \end{bmatrix} &= \begin{bmatrix}
                    \ddot{x_3} \\
                    \ddot{y_3} \\ 
                    \ddot{z_3}
                \end{bmatrix}
            \end{align}

            Because this problem is in three dimensions, there are actually 9 differential equations to solve in order to fully model the evolution of the position and velocity over time of each object.

        \subsection{Computational Model}
        
            \subsubsection{Finite Differences}

                Because computers can only store numbers of finite precision, approximations must be used in order to implement these differntial equations in a form that can be understood by the computer.  One such translation is by approximating the analytic derivatives and differential operators with \emph{finite differences}. In its most basic form, the derivative can be approximated as a ratio of differences in the form

                \begin{equation}
                    \frac{df}{dt} \approx \frac{f(t + \Delta t) - f(t)}{\Delta t}.
                \end{equation}

                This equation is known as the first-order-accurate, forward-finite-difference representation of the first-order derivative \textbf{[NEEDS CITATION]}.  There are similar representations for higher-order derivatives, more accurate approximations, and more accurate approximations of higher-order derivatives. Similarly, there are also \emph{central} and \emph{backward} finite-differences.

                Each of these finite-difference schemes can be represented via coefficients.  For example, the $m$-the derivative with accuracy $n$ has $2p + 1 = 2 \lfloor \frac{m + 1}{2} \rfloor - 1 + n$ central coefficients $a_{-p}, a_{-p + 1}, \ldots, a_{p - 1}, a_p$ defined such that

                \begin{equation}
                    \begin{bmatrix}
                        1         & 1             & \ldots & 1            & 1 \\
                        -p        & -p + 1        & \ldots & p - 1        & p \\
                        (-p)^2    & (-p + 1)^2    & \ldots & (p - 1)^2    & p^2 \\
                        \vdots    & \vdots        & \vdots & \vdots       & \vdots \\
                        (-p)^{2p} & (-p + 1)^{2p} & \ldots & (p - 1)^{2p} & p^{2p} 
                    \end{bmatrix}
                    \begin{bmatrix}
                        a_{-p} \\
                        a_{-p + 1} \\
                        \vdots \\
                        \vdots \\
                        a_p
                    \end{bmatrix} = 
                    \begin{bmatrix}
                        0 \\
                        \vdots \\
                        m! \\
                        \vdots \\
                        0
                    \end{bmatrix}
                \end{equation}

                where the only non-zero value on the right side of the equation is at the $m+1$-th index\textbf{[NEEDS CITATION]}.
                
                Once the derivative operators are discretized using finite-differences, there are different algorithms in which the differences are used to solve the differential equations.

            \subsubsection{Symplectic Integrators}

                While there are many methods of numerically solving differential equations (known as \emph{integrators}), one of the most important classes when considering a context such as orbital mechanics and oscillatory situations (i.e. Hamiltonian Systems) is that of \emph{symplectic} integrators.  This type of integrator is so important to computational physics because it conserves energy throughout time \textbf{[NEEDS CITATION]}.  Without energy conservation, the system becomes unphysical, which can lead to unnatural behavior such as a mass on a spring extending to infinity.  Two of the simplest symplectic integrators are the \emph{Semi-implicit Euler Method}\textbf{[NEEDS CITATION]}

                \begin{subequations} \label{eq:semiEuler}
                    \begin{equation}
                        v_{n+1} = v_n + a_n \Delta t
                    \end{equation}
                    \begin{equation}
                        x_{n + 1} = x_n + v_{n+1} \Delta t
                    \end{equation}
                \end{subequations}

                and the \emph{Velocity Verlet Algorithm}\textbf{[NEEDS CITATION]}

                \begin{subequations} \label{eq:verlet}
                    \begin{equation}
                        v_{n+1} = v_n + \frac{a_n + a_{n+1}}{2} \Delta t
                    \end{equation}
                    \begin{equation}
                        x_{n + 1} = x_n + v_{n} \Delta t + \frac{1}{2} a_n \Delta t^2
                    \end{equation}
                \end{subequations}

                The velocity-Verlet algorithm in C:

                \begin{lstlisting}[language=C]
void velocity_verlet(double *time, double *vel,
                     double pos, double acc, double dt) {
    *vel = 0.0;
    *time = 0.0;

    while (pos > 0) {
        *time += dt;
        pos += (*vel) * dt + 0.5 * acc * dt * dt;
        *vel += acc * dt;
    }
}               \end{lstlisting}

            \subsubsection{Parallel Computing}

                Because there are so many terms and equations, and each term is only needs the current step, this problem naturally lends itself to being parallelized\cite{parallelnbody}.

        \subsection{Numerical Analysis}

            \subsubsection{Discretization}

                Error associated with the size of the time-step is dictated by which specific finite-difference method is chosen.

            \subsubsection{Energy Drift}

                In the case of orbital mechanics, there is no loss of mechanical energy (ignoring the effects of general relativity; more on this later in section \ref{sec:relativity}).  Therefore, a constraint is imposed on the system that the total mechanical energy (kinetic energy + potential energy) of the system must remain constant at each point in time.  When using non-symplectic methods, this constraint is not adhered to which leads to \emph{energy drift}.

\pagebreak

    \section{Fluid Dynamics}

        \subsection{Mathematical Model}

            \begin{itemize}
                \item Navier-Stokes
            \end{itemize}

        \subsection{Computational Tools}

            \begin{itemize}
                \item ?
            \end{itemize}

        \subsection{Numerical Analysis}

            \begin{itemize}
                \item ?
            \end{itemize}

    \section{Projectile Motion with Drag}

        \subsection{Mathematical Model}

            \begin{itemize}
                \item linear drag
                \item quadratic Drag
                \item both -> system of nonlinear differential equations
            \end{itemize}

        \subsection{Computational Tools}

        \subsection{Numerical Analysis}

\chapter{Electromagnetism}

    \section{Fringing Electric Fields of Non-Ideal Capacitors}

        \subsection{Mathematical Model}

            \begin{itemize}
                \item maxwell's equations
            \end{itemize}

        \subsection{Computational Tools}

            \begin{itemize}
                \item relaxation method
            \end{itemize}

        \subsection{Numerical Analysis}

            \begin{itemize}
                \item spatial mesh
            \end{itemize}

    \section{Antenna Radiation Patterns}

        \subsection{Mathematical Model}

            \begin{itemize}
                \item maxwell's equations
            \end{itemize}

        \subsection{Computational Tools}

            \begin{itemize}
                \item 
            \end{itemize}

        \subsection{Numerical Analysis}

            \begin{itemize}
                \item 
            \end{itemize}

\chapter{Quantum Mechanics} \label{sec:quantum}

    \section{Quantum Chemistry}

        \subsection{Mathematical Model}

            \begin{itemize}
                \item The Schr{\"o}dinger Equation
                \item The Many-Body Schr{\"o}dinger Equation
            \end{itemize}

        \subsection{Computational Tools}

        \subsection{Numerical Analysis}

    \section{Quadrupole-Quadrupole Interactions in a BEC}

        \subsection{Mathematical Model}

            \begin{itemize}
                \item The Schr{\"o}dinger Equation
                \item The Many-Body Schr{\"o}dinger Equation
            \end{itemize}

        \subsection{Computational Tools}

        \subsection{Numerical Analysis}

\chapter{Relativity} \label{sec:relativity}

    \section{Mercury's Perihelion Shift}

    \section{Effects of Eccentricity}

    \section{Gravitational Wave Chirp}

    \section{Gravitational Waves}






    \section{Mathematical Model}

    \section{Computational Tools}

    \section{Numerical Analysis}

\chapter{Chaotic Systems}

    \section{Atmospheric Physics}

    \section{Forced Oscillators}

    \section{Mathematical Model}

    \section{Computational Tools}

    \section{Numerical Analysis}

\chapter{Conclusion}

\newpage

\printbibliography

\end{document}
\documentclass{beamer}
\usetheme{Madrid}
\title[Computational Physics]{A Survey of Computational Physics}
% \subtitle{}

\author[Chapman, Nathan]{Nathan Chapman\inst{1}}

\institute[CWU]
{
  \inst{1}
  Faculty of Physics\\
  Central Washington University
%   \and
%   \inst{2}%
%   Faculty of Chemistry\\
%   Very Famous University
}

\date[HPC 2024]{CS 530: High Performance Computing, Spring 2024}

\logo{\includegraphics[height=0.5cm]{images/cwu_logo.png}}

\begin{document}

\frame{\titlepage}

% \begin{frame}
%     \frametitle{Table of Contents}
%     \tableofcontents
% \end{frame}

\begin{frame}
\frametitle{Introduction}
    \hspace{0.25in} \underline{Pillars of Physics} \hfill \underline{Computational Applications}
    \begin{itemize}
        \item <1-> Classical Mechanics \phantom{  } \rightarrowfill \phantom{  } N-Body Simulations \& Fluid Dynamics
        \item <2-> Electromagnetism \phantom{  } \rightarrowfill \phantom{  } Fringing Fields \& Antenna Radiation
        \item <3-> Thermodynamics \phantom{  } \rightarrowfill \phantom{  } ???
        \item <4-> Quantum Mechanics \phantom{  } \rightarrowfill \phantom{  } Molecular Optimization \& Ultra-Cold Gases
        \item <5-> Relativity \phantom{  } \rightarrowfill \phantom{  } Mercury's Perihelion \& Black Hole Ray Tracing
    \end{itemize}
\end{frame}

\begin{frame}
    \frametitle{Classical Mechanics}
    \begin{block}{Remark}
        Given the position and velocity of and the forces acting on an object, the motion of that object is completely determined.
    \end{block}
    \begin{itemize}
        \item Everday physics can be completed described by Newton's 2nd Law.
              \begin{equation} \label{eq:newton}
                  \underbrace{\sum_i \mathbf{F}_i = m \ddot{\mathbf{r}}}_\text{Equation of Motion}
              \end{equation}
    \end{itemize}
\end{frame}

\begin{frame}
    \frametitle{Sample frame title}
    
    In this slide, some important text will be
    \alert{highlighted} because it's important.
    Please, don't abuse it.
    
    \begin{block}{Remark}
    Sample text
    \end{block}
    
    \begin{alertblock}{Important theorem}
    Sample text in red box
    \end{alertblock}
    
    \begin{examples}
    Sample text in green box. The title of the block is ``Examples".
    \end{examples}
\end{frame}

\begin{frame}
    \frametitle{Two-column slide}
    \begin{columns}
    \column{0.5\textwidth}
    This is a text in first column.
    $$E=mc^2$$
    \begin{itemize}
    \item First item
    \item Second item
    \end{itemize}
    
    \column{0.5\textwidth}
    This text will be in the second column
    and on a second thoughts, this is a nice looking
    layout in some cases.
    \end{columns}
\end{frame}

\end{document}
\documentclass{beamer}

\usepackage{parskip, setspace}
\setstretch{1.5}

\usepackage{biblatex}
\addbibresource{bib.bib}
% math formatting
\usepackage{amsmath, amsfonts, braket}
% \numberwithin{equation}{section}
% rich text
\usepackage{graphicx, caption, multimedia}
\usepackage{hyperref}
\usepackage{xcolor}
\hypersetup{
    colorlinks=true,
    linkcolor=black,  
    urlcolor=blue,
    citecolor=blue,
    pdftitle={A Survey of Computational Physics},
    pdfpagemode=FullScreen,
}

\usepackage{algorithm2e}

\usetheme{Madrid}
\title[Computational Physics]{Computational Physics}
% \subtitle{}

\author[Chapman, Nathan]{Nathan Chapman}

\institute[CWU]
{
  Faculty of Physics\\
  Central Washington University
%   \and
%   \inst{2}%
%   Faculty of Chemistry\\
%   Very Famous University
}

\date[HPC 2024]{CS 530: High Performance Computing, Spring 2024}

\logo{\includegraphics[height=0.5cm]{images/cwu_logo.png}}

\begin{document}

\frame{\titlepage}

\begin{frame}
    % \frametitle{<title>}

    \begin{block}{}
        \centering \Large Why do we need computers to do physics?
    \end{block}

\end{frame}

\begin{frame}
    \frametitle{An Illuminating Example - 2-Body Problem}
    \vspace{-0.75in}
    \begin{equation*}
        \vec{F}_{ij} = \frac{G m_i m_j}{r_{ij}^2} \hat{r} \qquad \qquad \sum_j \vec{F}_{ij} = m_i \ddot{\vec{r}}_i
    \end{equation*}

    \begin{columns}
        \column{0.5\textwidth}
        \begin{itemize}
            \item 2 masses
            \item 1 unique force
            \item 2 equations of motion in 3 dimensions
            \item 6 coupled 2nd-order ordinary differential equations
        \end{itemize}

        \column{0.5\textwidth}
        \movie[autostart,loop,height=1.5in,width=\textwidth]{}{Orbit5.gif}
    \end{columns}
\end{frame}

\begin{frame}
    \frametitle{An Illuminating Example - 3-Body Problem}
    \vspace{-0.75in}
    \begin{equation*}
        \vec{F}_{ij} = \frac{G m_i m_j}{r_{ij}^2} \hat{r} \qquad \qquad \sum_j \vec{F}_{ij} = m_i \ddot{\vec{r}}_i 
    \end{equation*}

    \begin{columns}
        \column{0.5\textwidth}
        \begin{itemize}
            \item 3 masses
            \item 3 unique forces
            \item 3 equations of motion in 3 dimensions
            \item 9 coupled 2nd-order ordinary differential equations
        \end{itemize}

        \column{0.5\textwidth}
        \movie[autostart,loop,height=1.5in,width=\textwidth]{}{3_body.gif}
    \end{columns}
\end{frame}

\begin{frame}
    \frametitle{An Illuminating Example - ``All''-Body Problem}
    \vspace{-0.75in}
    \begin{equation*}
        \vec{F}_{ij} = \frac{G m_i m_j}{r_{ij}^2} \hat{r} \qquad \qquad \sum_j \vec{F}_{ij} = m_i \ddot{\vec{r}}_i 
    \end{equation*}

    \begin{columns}
        \column{0.5\textwidth}
        \begin{itemize}
            \item 1.24 trillion masses
            \item $7.688\times 10^{23}$ unique forces
            \item 1.24 trillion equations of motion in 3 dimensions
            \item 3.72 trillion coupled 2nd-order ordinary differential equations
        \end{itemize}

        \column{0.5\textwidth}
        \href{https://youtu.be/JAyrpJCC_dw?si=eY7EbSKib7siokgG}{Cosmological N-Body Simulation}\cite{Heitmann_2021}
    \end{columns}
\end{frame}

% \begin{frame}
%     \frametitle{An Illuminating Example}
%     \begin{columns}
%         \column{0.5\textwidth}
%         \begin{itemize}
%             \item Direct algorithm
%             \item Velocity Verlet
%         \end{itemize}
        
%         \column{0.5\textwidth}
%         Algorithm psuedo-code goes
        
%         \end{columns}
% \end{frame}

% \begin{frame}
% \frametitle{Introduction}
%     \hspace{0.25in} \underline{Pillars of Physics} \hfill \underline{Computational Applications}
%     \begin{itemize}
%         \item <1-> Classical Mechanics \phantom{  } \rightarrowfill \phantom{  } N-Body Simulations \& Fluid Dynamics
%         \item <2-> Electromagnetism \phantom{  } \rightarrowfill \phantom{  } Fringing Fields \& Antenna Radiation
%         \item <3-> Thermodynamics \phantom{  } \rightarrowfill \phantom{  } ???
%         \item <4-> Quantum Mechanics \phantom{  } \rightarrowfill \phantom{  } Molecular Optimization \& Ultra-Cold Gases
%         \item <5-> Relativity \phantom{  } \rightarrowfill \phantom{  } Mercury's Perihelion \& Black Hole Ray Tracing
%     \end{itemize}
% \end{frame}

\begin{frame}
    \frametitle{Classical Mechanics}
    \begin{block}{Remark}
        Given the position and velocity of and the forces acting on an object, the motion of that object is completely determined.
    \end{block}
    \begin{itemize}
        \item Everday physics can be completed described by Newton's 2nd Law.
              \begin{equation} \label{eq:newton}
                  \underbrace{\sum_i \mathbf{F}_i = m \ddot{\mathbf{r}}}_\text{Equation of Motion}
              \end{equation}
    \end{itemize}
\end{frame}

% \begin{frame}
%     \frametitle{Sample frame title}
    
%     In this slide, some important text will be
%     \alert{highlighted} because it's important.
%     Please, don't abuse it.
    
%     \begin{block}{Remark}
%     Sample text
%     \end{block}
    
%     \begin{alertblock}{Important theorem}
%     Sample text in red box
%     \end{alertblock}
    
%     \begin{examples}
%     Sample text in green box. The title of the block is ``Examples".
%     \end{examples}
% \end{frame}

\nocite{*}
\begin{frame}[allowframebreaks]
\printbibliography
\end{frame}

\end{document}
\documentclass{article}
\usepackage[utf8]{inputenc}
\usepackage[margin=0.625in]{geometry}
\usepackage{parskip, setspace}
\setstretch{1.5}
\usepackage{amsmath, amsfonts}
%\numberwithin{equation}{subsection}
\usepackage{graphicx, caption, wrapfig}
\usepackage{hyperref}
\usepackage{multirow}

\usepackage{biblatex}
\addbibresource{bib.bib}

\renewcommand{\arraystretch}{1.5}

\title{\Large \vspace{-0.625in} CS 530: High-Performance Computing \\ Benchmarking Matrix Multiplication \vspace{-0.15in}}
\author{Nathan Chapman \\ {\normalsize Central Washington University}}
\date{\normalsize \vspace{-0.1in}\today}

\begin{document}

\maketitle

    \begin{center}
    Abstract \\
    
    \end{center}

\tableofcontents

\pagebreak

\section{Introduction}

    \begin{itemize}
        \item Things are easier to write with scripting languages, but run longer
        \item Things are harder to write with compiled languages, but run faster
        \item Most things are done with scripts, few but very fundamental algorithms are implemented with compiled algorithms 
        \item It's worth the time to write
        \item 
    \end{itemize}

\section{Methods}

\section{Results}

\section{Discussion}

\section{Conclusion}

\end{document}